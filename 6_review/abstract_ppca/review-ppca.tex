% ------------------------------------------------------------------------------
% SETUP ------------------------------------------------------------------------
% ------------------------------------------------------------------------------

\documentclass[12pt]{article}

\renewcommand{\labelitemi}{$-$}
\renewcommand{\labelitemii}{$\circ$}
\renewcommand{\labelitemiii}{$\circ$}

\newcommand{\topo}{\mathcal{T}}
\newcommand{\mani}{\mathcal{M}}
\newcommand{\N}{\mathbb{N}}
\newcommand{\R}{\mathbb{R}}
\newcommand{\RD}{\mathbb{R}^D}
\newcommand{\Rd}{\mathbb{R}^d}
\newcommand{\setN}{\{1, 2, ..., N\}}
\newcommand{\X}{\mathcal{X}}
\newcommand{\x}{\bm{x}}
\newcommand{\Y}{\mathcal{Y}}
\newcommand{\y}{\bm{y}}

\usepackage[a4paper,width = 150mm, top = 25mm, bottom = 25mm, 
bindingoffset = 6mm]{geometry}
\usepackage[utf8]{inputenc}
\usepackage[round, comma]{natbib}
\usepackage{url}
\usepackage[font = footnotesize]{caption}
\usepackage{subcaption}
\usepackage{csquotes} \MakeOuterQuote{"}
\usepackage{ragged2e}
\usepackage{array}
\usepackage{tabularx}
\usepackage{amsmath}
\usepackage{bm}
\usepackage{amssymb}
\usepackage{amsthm }
\usepackage{graphicx}
\usepackage{float}
\usepackage[export]{adjustbox}
\usepackage[table]{xcolor}
% \usepackage{tikz}
%   \usetikzlibrary{trees, arrows, decorations.pathmorphing, backgrounds, 
%   positioning, fit, petri, shapes}
\usepackage{algorithm}
\usepackage{algorithmic}
\usepackage{xcolor, listings}
\usepackage{textcomp}
\usepackage{fancyhdr}
\newcommand{\changefont}{%
    \fontsize{8}{11}\selectfont
}
\usepackage{refcount}
\usepackage[hang,flushmargin]{footmisc} 

\newenvironment{tight_itemize}{
\begin{itemize}
  \setlength{\itemsep}{0pt}
  \setlength{\parskip}{0pt}
}{\end{itemize}}

\pagestyle{fancy}
\fancyhead{}
\fancyhead[R]{\changefont{Probabilistic PCA: Review of Extended Abstract}}
\fancyfoot{}
\fancyfoot[R]{\thepage}
\setlength{\headheight}{14.5pt}
\setlength{\parindent}{0pt}
\interfootnotelinepenalty = 10000

% ------------------------------------------------------------------------------
% MAIN -------------------------------------------------------------------------
% ------------------------------------------------------------------------------

\usepackage{Sweave}
\begin{document}
\Sconcordance{concordance:review-ppca.tex:review-ppca.Rnw:%
1 74 1 1 0 120 1}


% FRONT PAGE -------------------------------------------------------------------
 
\begin{titlepage}
\begin{center}
    
\LARGE
Review of Extended Abstract
    
\vspace{0.5cm}
      
\rule{\textwidth}{1.5pt}
\LARGE 
\textbf{Probabilistic PCA}
\rule{\textwidth}{1.5pt}
   
\vspace{0.5cm}

\large
Presenter: Roxana Shaikh \\
Supervisor: Moritz Herrmann  \\
Reviewer: Lisa Wimmer \\

\vspace{0.5cm}

January $15^{th}$, 2021

\end{center}
\end{titlepage}

% REVIEW -----------------------------------------------------------------------

\pagenumbering{arabic}

\section*{Review}    

\subsubsection*{Overall impression}

The extended abstract contains many important aspects and shows you have 
already gained a solid understanding of PPCA.
In particular, properties and drawbacks of the presented methods are pointed out 
and used to motivate their respective extensions/modifications.
This helps to establish a thread of arguments and yields a convincing story 
line.
In places, however, your explanations are not perfectly clear to me yet 
(see below).
Keeping a stronger focus on the potential audience might lead to better 
accessibility for readers.

\subsubsection*{Stated objective}
The objective is stated clearly and explicitly.
It appears to me just a little biased in its current wording, making it sound as 
if PPCA dominated PCA in all aspects.

\subsubsection*{Proposed structure}
What structure is given seems coherent, but some points strike me as missing 
entirely (which may, of course, be due to the brevity of an abstract):

\begin{tight_itemize}
  \item Given that the context of this seminar is manifold learning, some 
  motivation on dimensionality reduction would seem in order. 
  \item In a report on PPCA I would expect a solid introduction of 
  standard PCA.
  \item \citet{tippingbishop1999} themselves, and many others (e.g., 
  \citet{burges2004}, \citet{bouveyronetal2011}), stress the similarity between 
  PPCA and factor analysis, so this seems worth mentioning in motivating PPCA.
  \item Some comparison on related work could help to put PPCA into context.
  \item There is no reference to the practical implementation part.
  \item You do not list a discussion.
  Many hints at strengths and drawbacks are given in the text, but a dedicated 
  chapter for synopsis could help to provide a clear structure.
\end{tight_itemize}

\subsubsection*{Content outline}

\begin{tight_itemize}
  \item \textbf{Objective.} I am not sure I fully understand what you describe 
  as the principal strength/drawback here.
  It seems to me you might refer to work by \citet{shlens2005}. 
  Suggesting a probabilistic model so users may incorporate prior 
  knowledge is not entirely convincing to me.
  Personally, I would rather follow \citet{tippingbishop1999} and motivate the 
  probabilistic extension mainly via inference options and comparability to 
  other probabilistic approaches, compatibility to Bayesian frameworks, and the 
  possibility to construct model mixtures.
  \item \textbf{Probabilistic PCA.} 
  I would caution against this introduction of PPCA in the context of non-linear 
  methods -- after all, it is still a linear technique (and non-linearity is 
  achieved, in a way, only by a mixture of these linear models 
  \citep{tippingbishop1999}).
  In the second sentence, you give a summary of PPCA which, I believe, is 
  inspired by \citet{bishop2006}. 
  While this is certainly correct, I am not certain it the most on-point 
  description. 
  At least to me, the meaning is not fully clear at first glance.
  Lastly, I would expect some reference to the generative nature of PPCA.
  \item \textbf{MLPCA.} The abstract does not mention the central role Gaussians 
  play, especially in the MLE part, but this will certainly be explained in the 
  actual report.
  \item \textbf{MPPCA.} Here, I miss the motivation for generating mixtures of 
  PPCA analyzers (you describe it as "inevitable", by which you perhaps mean 
  that it results as a logical next step).
  The second paragraph, I am afraid, I do not quite understand.
\end{tight_itemize}

\subsubsection*{Further comments}

The abstract contains several spelling mistakes. 
For the report, I would recommend a thorough spell check -- it would be a shame 
if otherwise well-researched work suffered from a sloppy impression that does 
not do its contents justice.

% BIBLIOGRAPHY -----------------------------------------------------------------

\RaggedRight
\bibliography{bibliography}
\bibliographystyle{dcu}
\newpage

% ------------------------------------------------------------------------------

\end{document}
